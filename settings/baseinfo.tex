\newcommand\Lname{Мілюхін}
\newcommand\Fname{Олександр}
\newcommand\Initials{О. П.}

\newcommand\Institute{Інститут комп'ютерних наук та інформаційних технологій}
\newcommand\Department{САПР}
\newcommand\Group{ПП-24}
\newcommand\Variant{12}

\usepackage{xstring}

% ця команда отримує позначення дисципліни та обирає приймача роботи.
% я написав сценарій, який визначає це позначення за шляхом до поточного каталогу
% й після ще кількох перевірок генерує шаблон звіту, де заповнене все, крім дати й теми.
% Звісно, не завжди такий шаблонний метод працює, але тоді можна використати \renewcommand
\newcommand{\Work}[1]{
\IfStrEqCase{#1}{
		{num}{
			\newcommand\Discipline{Чисельні методи}
			\IfStrEqCase{\Type}{
				{\Lab}{
					\newcommand\Instructor{Чумакевич В. В.}
				}
				{\Pract}{
					\newcommand\Instructor{<++>}
				}
			}
		}
		{db}{
			\newcommand\Discipline{Організація баз даних та знань}
			\IfStrEqCase{\Type}{
				{\Lab}{
					\newcommand\Instructor{Головацький Р. І.}
				}
				{\Pract}{
					\newcommand\Instructor{<++>}
				}
			}
		}
		{stat}{
			\newcommand\Discipline{Теорія ймовірності та математична статистика}
			\IfStrEqCase{\Type}{
				{\Lab}{
					\newcommand\Instructor{<++>}
				}
				{\Pract}{
					\newcommand\Instructor{<++>}
				}
			}
		}
		{os}{
			\newcommand\Discipline{Операційні системи}
			\IfStrEqCase{\Type}{
				{\Lab}{
					\newcommand\Instructor{Мокрицька О.В.}
				}
				{\Pract}{
					\newcommand\Instructor{<++>}
				}
			}
		}
		{web}{
			\newcommand\Discipline{Технології веброзробки та дизайну}
			\IfStrEqCase{\Type}{
				{\Lab}{
					\newcommand\Instructor{<++>}
				}
				{\Pract}{
					\newcommand\Instructor{<++>}
				}
			}
		}
	}
}
