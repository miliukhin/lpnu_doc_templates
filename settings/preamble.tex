%\documentclass[12pt,oneside]{extreport}
\documentclass[12pt,oneside]{extarticle}
\sloppy

\usepackage{multicol} % багатостовпцевий текст
\usepackage{indentfirst} % абзаци на початку секцій
\usepackage{subfiles} % пакет, що полегшує вкладення файлів

\usepackage[a4paper]{geometry}
\newcommand\Margins
{
	\newgeometry{
	top=1.5cm,
	bottom=2cm,
	right=2cm,
	left=2cm
	}
} % Встановлення потрібних відступів

\usepackage[utf8]{inputenc} % input encoding
\usepackage[T2A]{fontenc} % font encoding

% українізація документа
\usepackage[english,ukrainian]{babel}
%(Бібліоґрафія замість Bibliography і т. п.)

%	графіка	%
\usepackage{graphicx} % для додання картинок
%\graphicspath{{images/}}

\usepackage{tikz}
\usetikzlibrary
{
	calc,
	positioning,
	shapes.geometric,
	shapes.symbols,
	shapes.misc
}

%	Рисунки й таке інше	%
\usepackage{wrapfig} % для розміщення рисунків безпосередньо в тексті
\usepackage{caption} % розширення функціоналу підписів
\usepackage{subcaption} % підпідписи

% Пакети, пов'язані з лістингом
% --------------------------%
\usepackage{moreverb} % тюнінгований verbatim
\let\verbatiminput=\verbatimtabinput % табуляція
\def\verbatimtabsize{4\relax}
% --------------------------%

\newcommand\Lname{<++>}
\newcommand\Fname{<++>}
\newcommand\Initials{<++>}

\newcommand\Institute{<++>}
\newcommand\Department{<++>}
\newcommand\Group{<++>}
\newcommand\Variant{<++>}

\usepackage{xstring}

% ця команда отримує позначення дисципліни та обирає приймача роботи.
% я написав сценарій, який визначає це позначення за шляхом до поточного каталогу
% й після ще кількох перевірок генерує шаблон звіту, де заповнене все, крім дати й теми.
% Звісно, не завжди такий шаблонний метод працює, але тоді можна використати \renewcommand
\newcommand{\Work}[1]{
\IfStrEqCase{#1}{
		{<++>}{
			\newcommand\Discipline{<++>}
			\IfStrEqCase{\Type}{
				{\Lab}{
					\newcommand\Instructor{<++>}
				}
				{\Pract}{
					\newcommand\Instructor{<++>}
				}
			}
		}
	}
}

%\newenvironment{qanda}{\setlength{\parindent}{0pt}}{\bigskip}
%\newcommand{\Q}{\bigskip\bfseries Q: }
%\newcommand{\A}{\medskip\par\textbf{A:} \hphantom{2em}\normalfont}

\newcounter{question}
\setcounter{question}{0}

\newcommand{\question}[1]{\bfseries\item[\refstepcounter{question}\thequestion.] {#1}}
%\newcommand{\answer}[1]{\item[A\thequestion.] \normalfont#1}
\newcommand{\answer}[1]{\item[] \normalfont#1}

\newcommand\Lab{лабораторної роботи}
\newcommand\Pract{практичної роботи}
\newcommand\RGR{Розрахунково-графічна робота}
%\newcommand\dhu{Доповідь}
 % Q&A
